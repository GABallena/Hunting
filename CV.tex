\documentclass[11pt,a4paper,sans]{moderncv}

% Adjust margins and layout
\usepackage[a4paper, margin=1in, bottom=1in]{geometry} % Adjusted bottom margin for footer

% ModernCV themes
\moderncvstyle{banking}
\moderncvcolor{blue}
\nopagenumbers{} % Suppress page numbers

% Required packages for ModernCV
\usepackage{fontspec}       % Font support for XeLaTeX or LuaLaTeX
\usepackage{fontawesome5}   % FontAwesome icons
\usepackage{academicons}    % ORCID icons

% Fancyhdr for custom footer
\usepackage{fancyhdr}
\pagestyle{fancy}
\fancyhf{}
\fancyfoot[L]{\normalsize References available only upon request and if shortlisted} %Larger footer text
\setlength{\footskip}{50pt} % Push footer closer to the page's bottom edge
\renewcommand{\headrulewidth}{0pt} % Remove header line
\renewcommand{\footrulewidth}{0pt} % Remove footer line

% Adjust list spacing globally
\usepackage{enumitem}
\setlist[itemize]{topsep=0pt, itemsep=1pt, leftmargin=1.5em}

% Set fonts to Garamond
\setmainfont{EB Garamond}      % Ensure this is installed
\setsansfont{EB Garamond}      % Consistent sans-serif font
\setmonofont{Courier New}      % Monospace font

% Personal data
\name{Gerald Amiel}{Ballena}
\title{Bioinformatics Specialist}
\email{gmballena@up.edu.ph}
\social[github]{GABallena}
\social[linkedin]{gerald-amiel-ballena}
\extrainfo{\aiOrcid \ \href{https://orcid.org/0009-0000-8857-9755}{orcid.org/0009-0000-8857-9755}}

% Custom title section
\renewcommand{\makecvtitle}{
	\vspace*{-2em}
	\begin{center}
		{\Huge \textbf{Gerald Amiel Ballena}} \\[0.5em]
		\faEnvelope \ \href{mailto:gmballena@up.edu.ph}{gmballena@up.edu.ph} \quad
		\faGithub \ \href{https://github.com/GABallena}{github.com/GABallena} \quad
		\faLinkedin \ \href{https://linkedin.com/in/gerald-amiel-ballena}{linkedin.com/in/gerald-amiel-ballena} \\[0.5em]
		\faMapMarker \ Metro Manila, Philippines \quad
		\aiOrcid \ \href{https://orcid.org/0009-0000-8857-9755}{orcid.org/0009-0000-8857-9755}
	\end{center}
	\vspace{1.5em}
}

%----------------------------------------------------------------------------------
% Document Content
%----------------------------------------------------------------------------------
\begin{document}
	
	\makecvtitle
	
\section{Professional Summary}
Versatile bioinformatics specialist with  focused on general expertise in scalable workflows, computational biology, and machine learning. Skilled in developing scalable pipelines and implementing robust methodologies to tackle complex biological problems.



\section{Experience}
\textbf{2024 – Present} \hfill \textbf{Project Technical Specialist} \\
\textit{University of the Philippines, College of Public Health} \\[-1em]
\begin{itemize}
	\item Developed scalable bioinformatics workflows for high-throughput sequence analysis.
	\item Applied metagenomic profiling techniques to environmental datasets.
	\item Collaborated with cross-disciplinary teams on projects involving data visualization and computational modeling.
\end{itemize}



\section{Skills}

\subsection{Programming}
\cvitem{Languages}{Python, R, Bash, Perl}
\cvitem{Installation Toolkits}{BiocManager, upcxx}

\subsection{Bioinformatics Tools}
\cvitem{Metagenomics}{Kraken2, MetaPhlAn, HUMAnN}
\cvitem{Assembly \& Binning}{MEGAHIT, METAWRAP, CheckM}
\cvitem{Genomics}{CLC Workbench, Roary, MinHash}
\cvdoubleitem{Quality Control}{FastQC, BUSCO, QUAST}{Trimming}{Trimmomatic, Sickle, Cutadapt}
\cvitem{Phylogenetics}{RaxML, FastTree, IQ-TREE, phyML, BEAST}
\cvitem{Gene Ontology}{KEGG, GOseq}
\cvitem{Annotation}{Prokka, ShortBRED, EggNOG-mapper}
\cvitem{Others}{DESeq2, BBMap Suite, anvio-8, SPAdes}

\subsection{Data Analysis \& Visualization}
\cvitem{Machine Learning}{scikit-learn, caret (R)}
\cvitem{Visualization}{ggplot2, Plotly, Krona, Shiny, Seaborn, Tableau}

\subsection{Technical Skills}
\cvitem{Version Control}{Git, GitHub}
\cvitem{High-Performance Computing}{Slurm}
\cvitem{Workflow Automation}{Snakemake, Conda, YAML (config files)}
\cvitem{Virtualization}{Docker, WSL2}

\subsection{Other Tools}
\cvitem{Technical Writing}{Overleaf, TeXStudio}
\cvitem{Collaboration}{Thunderbird, Notion, MS Word (mailing lists)}
\cvitem{Documentation}{TeXStudio, Jupyter Notebooks}


	
\section{Certifications and Relevant Coursework}

\subsection{Certifications}
\cvitem{AI Fundamentals}{\dotfill Nov 2024}
\cvitem{Data Literacy}{\dotfill Nov 2024}

\subsection{Fundamentals Courses}
\cvitem{Introduction to Data Engineering}{\dotfill Nov 2024}

\subsection{Intermediate-Level Courses}
\cvitem{Intermediate R}{\dotfill Nov 2024}
\cvitem{Introduction to Bioconductor in R}{\dotfill Nov 2024}
\cvitem{RNA-Seq with Bioconductor in R}{\dotfill Nov 2024}
\cvitem{Differential Expression Analysis with limma in R}{\dotfill Nov 2024}
\cvitem{ChIP-Seq with Bioconductor in R}{\dotfill Nov 2024}
\cvitem{Analyzing Genomic Data in R}{\dotfill Nov 2024}

\subsection{Advanced-Level Courses}
\cvitem{Hyperparameter Tuning in R}{\dotfill Nov 2024}
\cvitem{Designing Machine Learning Workflows in Python}{\dotfill Nov 2024}

\subsection{Ongoing Courses}
\cvitem{Ensemble Methods in Python}{Bagging, Boosting, Stacking}
\cvitem{Visualizing Geospatial Data in R}{Visualizing complex spatial datasets for actionable insights.}

\subsection{Assessments}
\cvitem{Azure Fundamentals}{Advanced | Score: 180 | Percentile: 99th}
\cvitem{Data Storytelling}{Advanced | Score: 200 | Percentile: 99th}


\section{Projects}

\cvitem{}{For more details on public projects, visit {GitHub Documentation.pdf}. \\
	\textbf{Note} Some repositories including Documentation repo access is currently restricted due to NDA obligations.}


\subsection{Scripts Repositories}

\cvitem{Side repo}{Scripts for general-purpose tasks, including package management and workflow setup.}
\cvitem{Experimental Repository}{Exploratory scripts focused on innovative computational techniques.}
\cvitem{Health repo}{Tools and scripts aimed at health-related data analysis and personal wellness tracking.}
\cvitem{Finance repo}{\textbf{WIP}; developing scripts for personal finance management.}
\cvitem{Project4 repo}{Scripts designed for NGS analyses and workflow automation; details kept private due to project confidentiality.}
\cvitem{Kitchen repo}{We cookin' some unconventional ideas here.}

\subsection{Documentation Repositories}

\cvitem{Documentation repo}{Explanation of every script I've ever written; written in a non-technical tone for non-technical audiences.}
\cvitem{Confidential repo}{LaTeX-compiled reports and documentation of my work as PTS I; restricted access due to confidentiality.}
\cvitem{Hunting repo}{Includes this CV, my resume, associated \texttt{.tex} files, and certifications; tailored for job hunting.}



\subsection{Key Projects}

\cvitem{Bioinformatics Workflow Development}{
	Designed and implemented modular pipelines for omics data analysis, emphasizing quality control, data assembly, and scalable workflows. Focused on creating reproducible methodologies applicable across diverse datasets.}

\cvitem{Spatial Data Analysis}{
	Utilized QGIS, R and geospatial libraries to visualize spatial patterns in biological datasets. Developed custom visualization tools for integrating geospatial and genomic data in ecosystem-level studies.}

\cvitem{Public Health Data Exploration}{
	Developed computational approaches for microbial profiling and identifying markers of interest in public health datasets. Leveraged metagenomic techniques to enable actionable insights for research studies.}


\cvitem{Development of K-mer Analysis and Statistical Workflow}{
	\begin{itemize}
		\item \textbf{Objective:} Build an advanced pipeline for k-mer generation, variance analysis, and distribution modeling.
		\item \textbf{Details:} Focused on creating statistical workflows to evaluate parametric and non-parametric fits, automating evaluations with custom Python scripts.
		\item \textbf{Tools \& Technologies:} Jellyfish, Python, Snakemake, Conda.
	\end{itemize}
}

\cvitem{High-Accuracy Sequence Alignment Workflow}{
	\begin{itemize}
		\item \textbf{Objective:} Create a modular workflow for high-identity sequence alignment with validation across multiple tools.
		\item \textbf{Details:} Integrated tools like Bowtie2, BWA, and Minimap2 with standardized outputs to enable comparative analysis.
		\item \textbf{Tools \& Technologies:} Bowtie2, BWA, Minimap2, KMA, Python, Snakemake.
	\end{itemize}
}

\cvitem{Shannon Entropy Analysis of K-mers and Taxonomic Profiles}{
	\begin{itemize}
		\item \textbf{Objective:} Quantify sequence complexity and diversity using entropy-based metrics.
		\item \textbf{Details:} Developed workflows to calculate and integrate Shannon entropy metrics for assessing genomic diversity and data quality.
		\item \textbf{Tools \& Technologies:} Jellyfish, Kraken2, Python, Snakemake.
	\end{itemize}
}

\cvitem{Comprehensive Workflow for Contaminant Removal and SCG Validation}{
	\begin{itemize}
		\item \textbf{Objective:} Design a robust pipeline for contaminant filtering and SCG validation.
		\item \textbf{Details:} Automated filtering using k-mer-based methods and validated results with SCG retention analysis via BUSCO.
		\item \textbf{Tools \& Technologies:} BUSCO, Jellyfish, Python, Snakemake, Conda.
	\end{itemize}
}

\cvitem{Pipeline Automation and Dependency Management}{
	\begin{itemize}
		\item \textbf{Objective:} Streamline bioinformatics workflows with automated dependency checks and alias creation.
		\item \textbf{Details:} Developed:
		\begin{itemize}
			\item \texttt{bioconda\_search.py} to identify relevant bioinformatics tools.
			\item \texttt{create\_aliases.py} to automate Bash alias generation.
			\item \texttt{check\_dependencies.py} and \texttt{append\_new.py} for managing and updating YAML files.
		\end{itemize}
		\item \textbf{Tools \& Technologies:} Python, YAML, Conda, Bioconda.
	\end{itemize}
}

\cvitem{Version Management and Workflow Validation}{
	\begin{itemize}
		\item \textbf{Objective:} Ensure reproducibility in bioinformatics workflows.
		\item \textbf{Details:} \texttt{versioncheck.bash} validates installed tool versions against expected configurations, improving workflow consistency.
		\item \textbf{Tools \& Technologies:} Bash, Python, YAML.
	\end{itemize}
}

\subsection{Ongoing Coursework}
Ensemble Methods in Python
Visualizing Geospatial Data in R


	
\section{Languages}
\cvitemwithcomment{English}{Fluent}{Professional proficiency.}
\cvitemwithcomment{Tagalog}{Native}{Conversational}
\cvitemwithcomment{Ilocano}{Can understand}{Wernicke's area skill issue.}

\subsection{Computer Languages (For Fun)}
\cvitemwithcomment{Python}{Fluent}{Fluent enough for advanced bioinformatics.}
\cvitemwithcomment{R}{Advanced}{Best data visualization tool out there (currently).}
\cvitemwithcomment{Bash}{Intermediate}{Go-to shell scripting language.}
\cvitemwithcomment{Perl}{Fair}{Would rather "speak" in Bash.}
\cvitemwithcomment{C++}{Basic}{As fluent as someone who hasn't used it in a decade.}

\section{References}
\textbf{Available upon request \& only if already shortlisted.}


\end{document}

% Add at the very end of the document
