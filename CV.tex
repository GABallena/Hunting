\documentclass[11pt,a4paper,sans]{moderncv}

% Adjust margins and layout
\usepackage[a4paper, margin=1in, bottom=1in]{geometry} % Adjusted bottom margin for footer

% ModernCV themes
\moderncvstyle{banking}
\moderncvcolor{blue}
\nopagenumbers{} % Suppress page numbers

% Required packages for ModernCV
\usepackage{fontspec}       % Font support for XeLaTeX or LuaLaTeX
\usepackage{fontawesome5}   % FontAwesome icons
\usepackage{academicons}    % ORCID icons

% Fancyhdr for custom footer
\usepackage{fancyhdr}
\pagestyle{fancy}
\fancyhf{}
\fancyfoot[L]{\normalsize References available only upon request and if shortlisted} %Larger footer text
\setlength{\footskip}{50pt} % Push footer closer to the page's bottom edge
\renewcommand{\headrulewidth}{0pt} % Remove header line
\renewcommand{\footrulewidth}{0pt} % Remove footer line

% Adjust list spacing globally
\usepackage{enumitem}
\setlist[itemize]{topsep=0pt, itemsep=1pt, leftmargin=1.5em}

% Set fonts to Garamond
\setmainfont{EB Garamond}      % Ensure this is installed
\setsansfont{EB Garamond}      % Consistent sans-serif font
\setmonofont{Courier New}      % Monospace font

% Personal data
\name{Gerald Amiel}{Ballena}
\title{Bioinformatics Specialist}
\email{gmballena@up.edu.ph}
\social[github]{GABallena}
\social[linkedin]{gerald-amiel-ballena}
\extrainfo{\aiOrcid \ \href{https://orcid.org/0009-0000-8857-9755}{orcid.org/0009-0000-8857-9755}}

% Custom title section
\renewcommand{\makecvtitle}{
	\vspace*{-2em}
	\begin{center}
		{\Huge \textbf{Gerald Amiel Ballena}} \\[0.5em]
		\faEnvelope \ \href{mailto:gmballena@up.edu.ph}{gmballena@up.edu.ph} \quad
		\faGithub \ \href{https://github.com/GABallena}{github.com/GABallena} \quad
		\faLinkedin \ \href{https://linkedin.com/in/gerald-amiel-ballena}{linkedin.com/in/gerald-amiel-ballena} \\[0.5em]
		\faMapMarker \ Metro Manila, Philippines \quad
		\aiOrcid \ \href{https://orcid.org/0009-0000-8857-9755}{orcid.org/0009-0000-8857-9755}
	\end{center}
	\vspace{1.5em}
}

%----------------------------------------------------------------------------------
% Document Content
%----------------------------------------------------------------------------------
\begin{document}
	
	\makecvtitle
	
\section{Professional Summary}
Versatile bioinformatics specialist with extensive experience in developing scalable workflows, computational biology, and machine learning. Successfully processed multi-terabyte datasets, streamlined bioinformatics pipelines to enhance reproducibility, and enabled actionable insights in environmental and public health projects.

\section{Experience}
\textbf{2024 – Present} \hfill \textbf{Project Technical Specialist} \\
\textit{University of the Philippines, College of Public Health} \\[-1em]
\begin{itemize}
	\item Developed scalable bioinformatics workflows for high-throughput sequence analysis (2.5 Terabytes each).
	\item Automated data preprocessing workflows using Snakemake, reducing manual load by 90\% when data is available while also enhancing reproducibility and removing human error.
	\item Collaborated with cross-disciplinary teams on projects involving public health, microbiology, and science of environmental engineering.
	\item Enhanced data analysis pipelines, leading to actionable insights for surveillance and public health research.
\end{itemize}

\section{Education}
\subsection{University of the Philippines Diliman}
\cvitem{Graduated}{July 2022}
\cvitem{Thesis}{In silico assessment of the association of pathogenicity and metal-resistance potential of \textit{Fusarium} spp.\dotfill \href{https://www.biorxiv.org/content/10.1101/2022.10.12.511937v1}{Pre-print Link}}
\cvitem{Accomplishments}{DOST ASTHRDP-Scholarship \\ Calculated Effective GPA (MS): 1.72}

\subsection{University of the Philippines Baguio}
\cvitem{Graduated}{June 2018} 
\cvitem{Thesis}{\textbf{Bioelectrocatalysis by Novel Electrogenic Alkaliphilic Bacteria \textit{Bacillus} sp. BAB-3442 Using Dual-Chambered Microbial Fuel Cell} \dotfill Poster Presentation (\texttt{PSM 47})}
\cvitem{Accomplishments}{Advanced Placement Exam: Advanced Algebra}

\subsection{Philippine Science High School CAR Campus}
\cvitem{Graduated}{March 2014}
\cvitem{Accomplishments}{Focused on STEM curriculum with a strong emphasis on research and scientific inquiry.}


\section{Skills}
\subsection{Technical Skills}
\cvitem{Programming Languages}{Python (Fluent), R (Advanced), Bash (Intermediate), Perl \& BioPerl (Intermediate), C++ (Fair).}
\cvitem{Bioinformatics Tools}{Kraken2, MetaPhlAn, MEGAHIT, Snakemake, Prokka, Jellyfish, BUSCO, and others.}
\cvitem{Workflow Automation}{Snakemake, Conda, YAML (Configuration Files).}
\cvitem{Data Visualization}{ggplot2, Plotly, matplotlib, QGIS.}
\cvitem{High-Performance Computing}{Slurm, Docker.}


\subsection{Soft Skills}
\cvitem{Collaboration}{Proficient in leading cross-disciplinary projects and fostering effective team collaboration.}
\cvitem{Technical Writing}{Skilled in preparing technical reports and documentation using TeX tools such as TeXStudio and Overleaf.}
\cvitem{Problem-Solving}{Adept at diagnosing and optimizing bioinformatics pipelines to enhance efficiency, reproducibility, and both statistical and scientific robustness.}



\section{Certifications and Relevant Coursework}
\subsection{Certifications}
\cvitem{Fundamental}{AI Fundamentals \dotfill  2024}
\cvitem{Fundamental}{Data Literacy  \dotfill 2024}

\subsection{Relevant Courses}
\cvitem{Introductory}{Introduction to Data Engineering \dotfill 2020}
\cvitem{Introductory}{COVID-19 Contact Tracing \dotfill 2020}
\cvitem{Introductory}{Mind Control: Managing Your Mental Health During COVID-19 \dotfill 2020}
\cvitem{Introductory}{COVID-19: What You Need to Know \dotfill 2020}
\cvitem{Introductory}{Essential Epidemiologic Tools for Public Health Practice \dotfill 2020}
\cvitem{Intermediate}{Biostatistics in Public Health \dotfill 2020}
\begin{itemize}
	\item Summary Statistics in Public Health
	\item Hypothesis Testing In Public Health
	\item Simple Regression Analysis in Public Health
\end{itemize}
\cvitem{Intermediate}{Genomic Analysis track \dotfill 2024}
\begin{itemize}
	\item Bioconductor in R
	\item RNA-Seq with Bioconductor
	\item Differential Expression Analysis with \texttt{limma}
	\item CHIP-Seq with Bioconductor in \texttt{R}
\end{itemize}  
\cvitem{Advanced}{Hyperparameter Tuning in R, Designing Machine Learning Workflows in Python \dotfill  2024}
\cvitem{Advanced}{Bioinformatics via Coursera \dotfill 2024}
(\textit{University of California, San Diego}) 
\begin{itemize}
	\item Finding Hidden Messages (with Honors)
\end{itemize}





\subsection{Professional Development}
\begin{itemize}
	\item Ensemble Methods in Python (Bagging, Boosting, Stacking)
	\item Visualizing Geospatial Data in R
	\item Introduction to AWS
	\item Principles, Statistical and Computational Tools for Reproducible Data Science
\end{itemize}


\section{Scripts and Workflows}

\cvitem{Key Pipelines and Workflows}{
	Developed and implemented scalable workflows and pipelines using Snakemake, Python, and Bash for metagenomic analysis, diversity profiling, and bioinformatics tool management. Highlights include:
	\begin{itemize}
		\item \textbf{Metagenomic Analysis Pipeline:} Automated workflows for trimming, taxonomic profiling (\textbf{Kraken2}, \textbf{Bracken}), and diversity calculations (\textbf{scikit-bio}), ensuring high reproducibility and scalability.
		\item \textbf{Comprehensive Binning Workflow:} Designed workflows for assembly, binning (\textbf{MetaWRAP}, \textbf{MEGAHIT}), and MAG validation (\textbf{CheckM2}), streamlining large-scale metagenomic projects.
		\item \textbf{Plasmid and ARG Analysis:} Built pipelines for plasmid detection and antimicrobial resistance profiling using \textbf{metaSPAdes}, \textbf{PlasmidFinder}, and \textbf{RGI}.
		\item \textbf{Diversity Analysis:} Developed Python-based scripts and R workflows for alpha/beta diversity metrics (Shannon, Chao1, Bray-Curtis) and visualizations using \textbf{ggplot2}.
	\end{itemize}
}

\cvitem{Automation and Custom Tools}{
	Created tools and workflows for parameter optimization, repository mining, and bioinformatics tool management:
	\begin{itemize}
		\item \textbf{Randomized Parameter Testing:} Automated preprocessing parameter exploration for tools like \textbf{Trimmomatic}, \textbf{Cutadapt}, and \textbf{fastp}, enabling systematic optimization.
		\item \textbf{Bioconda Repository Mining:} Developed a Python-based scraper to extract and filter bioinformatics tools for metagenomics and AMR research.
		\item \textbf{General Bootstrapping Workflow:} Automated sampling of paired-end reads for diversity and functional analyses using \textbf{seqtk}.
		\item \textbf{Tool Management:} Streamlined dependency discovery, YAML updates, and Conda-based environment management for reproducible pipelines.
	\end{itemize}
}

\cvitem{Advanced Analysis and Statistical Workflows}{
	Designed workflows for k-mer analysis, contaminant detection, and statistical evaluations:
	\begin{itemize}
		\item \textbf{K-mer Analysis:} Automated frequency distribution fitting, entropy calculations, and alignment validation for metagenomic datasets (\textbf{Jellyfish}, \textbf{MASH}).
		\item \textbf{Contaminant Filtering:} Built pipelines for k-mer mapping and statistical testing against known contaminant databases (\textbf{UniVec}, \textbf{PhiX}, \textbf{KMA}).
		\item \textbf{Visualization Pipeline:} Developed R-based workflows for ridgeline and violin plots, NMDS, and heatmaps to visualize diversity and taxonomic profiles.
	\end{itemize}
}







\end{document}

% Add at the very end of the document
