\documentclass[11pt,a4paper,sans]{moderncv}

% Adjust margins and layout
\usepackage[a4paper, margin=1in, bottom=1in]{geometry}

% ModernCV themes
\moderncvstyle{banking}
\moderncvcolor{blue}
\nopagenumbers{}

% Personal data
\name{Gerald Amiel}{Ballena}
\title{Bioinformatics Specialist}
\email{gmballena@up.edu.ph}
\social[github]{GABallena}
\social[linkedin]{gerald-amiel-ballena}
\extrainfo{\aiOrcid \ \href{https://orcid.org/0009-0000-8857-9755}{orcid.org/0009-0000-8857-9755}}

% Custom title section
\renewcommand{\makecvtitle}{
	\vspace*{-2em}
	\begin{center}
		{\Huge \textbf{Gerald Amiel Ballena}} \\[0.5em]
		\faEnvelope \ \href{mailto:gmballena@up.edu.ph}{gmballena@up.edu.ph} \quad
		\faGithub \ \href{https://github.com/GABallena}{github.com/GABallena} \quad
		\faLinkedin \ \href{https://linkedin.com/in/gerald-amiel-ballena}{linkedin.com/in/gerald-amiel-ballena} \\[0.5em]
		\faMapMarker \ Metro Manila, Philippines \quad
		\aiOrcid \ \href{https://orcid.org/0009-0000-8857-9755}{orcid.org/0009-0000-8857-9755}
	\end{center}
	\vspace{1.5em}
}

\begin{document}
	
	\makecvtitle
	
	\section{Professional Summary}
	Software Engineer and Bioinformatics Specialist with extensive experience developing computational pipelines for sequencing data and designing reproducible workflows. Proficient in Python, data visualization, and high-throughput data analysis, with a demonstrated ability to work on interdisciplinary bioinformatics projects, including 16S RNA sequencing.
	
	\section{Experience}
	\textbf{2024 – Present} \hfill \textbf{Project Technical Specialist} \\
	\textit{University of the Philippines, College of Public Health} \\[-1em]
	\begin{itemize}
		\item Developed scalable pipelines for high-throughput sequencing data (2.5 TB each), including preprocessing, assembly, and taxonomic profiling using \textbf{Snakemake}.
		\item Designed Python scripts for analyzing sequencing data, automating tasks, and generating insightful visualizations.
		\item Collaborated with interdisciplinary teams, contributing to public health surveillance and environmental microbiology projects.
		\item Enhanced reproducibility by implementing automated documentation and detailed logging systems.
		\item Applied statistical analyses to sequencing data using \textbf{pandas} and \textbf{scikit-bio}.
	\end{itemize}
	
	\textbf{2022 – 2024} \hfill \textbf{Graduate Researcher} \\
	\textit{University of the Philippines Diliman} \\[-1em]
	\begin{itemize}
		\item Built bioinformatics workflows for metagenomic analysis, including k-mer profiling, 16S RNA analysis, and antimicrobial resistance (AMR) gene prediction.
		\item Implemented automated diversity profiling using \textbf{Python} and \textbf{R} for public health studies.
		\item Developed robust pipelines for taxonomic profiling (\textbf{Kraken2}, \textbf{Bracken}) and genome assembly (\textbf{MEGAHIT}, \textbf{metaSPAdes}).
		\item Conducted exploratory data analyses and visualized results using \textbf{matplotlib} and \textbf{ggplot2}.
	\end{itemize}
	
	\section{Education}
	\subsection{University of the Philippines Diliman}
	\cvitem{Graduated}{July 2022}
	\cvitem{Thesis}{In silico assessment of the association of pathogenicity and metal-resistance potential of \textit{Fusarium} spp.\dotfill \href{https://www.biorxiv.org/content/10.1101/2022.10.12.511937v1}{Pre-print Link}}
	\cvitem{GPA}{1.72 (DOST ASTHRDP Scholar)}
	
	\subsection{University of the Philippines Baguio}
	\cvitem{Graduated}{June 2018} 
	\cvitem{Thesis}{\textbf{Bioelectrocatalysis by Novel Electrogenic Alkaliphilic Bacteria \textit{Bacillus} sp. BAB-3442 Using Dual-Chambered Microbial Fuel Cell}}
	\cvitem{Achievements}{Advanced Placement in Advanced Algebra}
	
	\section{Technical Skills}
	\subsection{Programming Languages}
	\cvitem{}{Python (Advanced), R (Intermediate), Bash (Intermediate), Perl (Intermediate)}
	
	\subsection{Bioinformatics Expertise}
	\cvitem{}{Taxonomic profiling (\textbf{Kraken2}, \textbf{Bracken}), genome assembly (\textbf{metaSPAdes}, \textbf{MEGAHIT}), k-mer profiling (\textbf{Jellyfish}, \textbf{MASH}), antimicrobial resistance gene detection (\textbf{RGI})}
	
	\subsection{Workflow Automation}
	\cvitem{}{Snakemake, Conda, Docker, Slurm}
	
	\subsection{Data Visualization and Analysis}
	\cvitem{}{\textbf{pandas}, \textbf{scikit-learn}, \textbf{matplotlib}, \textbf{ggplot2}, \textbf{QGIS}}
	
	\section{Key Projects}
	\cvitem{16S RNA Analysis Pipeline}{
		\begin{itemize}
			\item Designed pipelines for 16S RNA sequencing data analysis, focusing on diversity profiling and taxonomic classification.
			\item Automated read preprocessing and quality filtering using \textbf{Trimmomatic} and \textbf{Cutadapt}.
			\item Applied \textbf{pandas} and \textbf{scikit-bio} for data manipulation, statistical testing, and entropy calculations.
		\end{itemize}
	}
	
	\cvitem{Bioinformatics Tool Management}{
		\begin{itemize}
			\item Developed a Python scraper to automate dependency discovery and version updates for bioinformatics tools from Bioconda.
			\item Managed YAML-based environments for streamlined tool installation and reproducibility.
		\end{itemize}
	}
	
	\cvitem{Visualization Pipelines}{
		\begin{itemize}
			\item Generated histograms, violin plots, and ridgeline plots to visualize diversity metrics using \textbf{ggplot2} and \textbf{matplotlib}.
			\item Created heatmaps for beta diversity metrics and non-metric multidimensional scaling (NMDS) plots.
		\end{itemize}
	}
	
	\section{Certifications}
	\cvitem{Data Science and Bioinformatics}{Completed advanced courses on machine learning workflows, Bioconductor, and differential expression analysis.}
	
	\section{Soft Skills}
	\cvitem{Collaboration}{Proven ability to work effectively in cross-functional teams.}
	\cvitem{Problem-Solving}{Adept at debugging and optimizing computational pipelines.}
	\cvitem{Communication}{Skilled in presenting technical findings to both scientific and non-technical audiences.}
	
\end{document}
